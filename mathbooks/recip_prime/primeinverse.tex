\documentclass{jsarticle}
%ƒpƒbƒP[ƒW
%\documentclass[twocolumn]{jsarticle}
\usepackage[dvipdfm]{graphicx}
\usepackage{graphicx}
\usepackage{amsmath,amssymb}
\usepackage{fancybox}
\usepackage{ascmac}
\usepackage{amsthm}
\theoremstyle{definition}
\newtheorem{theorem}{’藝}
%memo:\newtheorem{ƒRƒ}ƒ“ƒh–Œ}{o‚µ‚œ‚¢•¶Žš}
\newtheorem*{theorem*}{’藝}
\newtheorem{definition}[theorem]{’è‹`}
\newtheorem*{definition*}{’è‹`}
\newtheorem{proposition}{–œ‘è}
\newtheorem*{proposition*}{–œ‘è}
\newtheorem{lemma}{•â‘è}
\newtheorem*{lemma*}{•â‘è}
\newtheorem{corollary}{Œn}
\newtheorem{example}{—á}
\newtheorem*{example*}{—á}
\newtheorem*{remark*}{’ˆÓ}
\newtheorem*{claim*}{Claim}
%\usepackage{mathptmx}
%
%ƒy[ƒW”ԍ†íœƒRƒ}ƒ“ƒh
%\pagestyle{empty}
%
%‘è–ÚƒR\ƒ}ƒ“ƒh
%\newcommand{\remark}{\textbf{’ˆÓF}}
%\newcommand{\review}{\textbf{•œK}F}
%\newcommand{\notation}{\textbf{‹L†F}}
%\newcommand{\memo}{\textbf{ƒƒ‚F}}
\newcommand{\al}{\alpha}
%alpha‚̏ȗªƒRƒ}ƒ“ƒh
%
%ˆÈ‰º—ªŽšƒRƒ}ƒ“ƒh
\newcommand{\p}{\mathfrak{p}}
\newcommand{\q}{\mathfrak{q}}
\newcommand{\N}{\mathbb{N}}
\newcommand{\Z}{\mathbb{Z}}
\newcommand{\Q}{\mathbb{Q}}
\newcommand{\R}{\mathbb{R}}
\newcommand{\C}{\mathbb{C}}
\newcommand{\F}{\mathbb{F}}
%
%textstyle
\setlength{\topmargin}{0pt}
\iftombow
\addtolength{\topmargin}{-1in}
\else
\addtolength{\topmargin}{-1truein}
\fi
%\setlength{\oddsidemargin}{-2cm}
%\setlength{\textwidth}{570pt}
\setlength{\footskip}{30pt}
\setlength{\textheight}{740pt}
\begin{document}
%%%%%%%%%%%%%%%%%%%%%%%%%%%%%%
\begin{center}
\Large{\shadowbox{‘f”‚Ì‹t”‚Ì•sŽv‹c}}
\end{center}
%%%%%%%%%%%%%%%%%%%%%%%%%%%%%%

¬ŠwZ‚Å‚Í
\[
\frac{1}{2}=0.5 ,\quad \frac{1}{3}=0.333 \cdots ,\quad \frac{1}{5}=0.2
\]
‚È‚Ç‚Ì‚æ‚€‚É•ª”‚ð¬”‚É’Œ‚·‚Æ‚¢‚€‚±‚Æ‚ð‚æ‚­‚â‚Á‚Ä‚¢‚œ‚ÆŽv‚¢‚Ü‚·.
‚Æ‚±‚낪, ’†ŠwZ‚ł́u”’lŒvŽZ‚Í•ª”‚Å‚Æ‚ß‚é‚æ‚€‚Ɂv‚Æ‹³‚í‚Á‚Ĉȗˆ, ”Šw‚̐¢ŠE‚ł͏¬”‚Í‚ ‚Ü‚èŠç‚ðo‚³‚È‚­‚È‚è‚Ü‚µ‚œ.
u‚µ‚å‚€‚·‚€v‚È‚Ÿ‚¯‚ɔނç‚͐”Šw‚̐¢ŠE‚ÅŒšg‚ð‹·‚­‚µ‚Ä‚¢‚é‚ÆŽv‚¢‚Ü‚·.
‚»‚ñ‚ȏ¬”‚Å‚·‚ª, ­‚µ–Ú‚ðŒü‚¯‚é‚Æ–Ê”’‚¢¢ŠE‚ªŒ©‚Š‚Ä‚«‚Ü‚·.
‚Å‚Í, –â‘è‚Å‚·.
\[
\frac{1}{12377} ‚̏¬”‘æ6193Œ…–Ú‚Ì’l‚ð‹‚ß‚æ.
\]

–ˆ”N“Œ–k‘åŠw—Šw•””Šw‰È‚̃I[ƒvƒ“ƒLƒƒƒ“ƒpƒX‚ł̓Cƒxƒ“ƒg‚Æ‚µ‚Đ”ŠwƒNƒCƒY‚ðs‚Á‚Ä‚¢‚Ü‚·.
‰ð‚¯‚œ‚çƒWƒ…[ƒX‚ª‚à‚ç‚Š‚é‚Æ‚¢‚€‚²–J”ü‚‚«‚Å‚·.
ã‚Ì–â‘è‚͍•–ؐ搶i•–Ø‚³‚ñj‚ª2009”N‚ɏo‘肳‚ê‚œ‚à‚Ì‚Å‚·(’Žß‚É‚ ‚éURL‚©‚çƒ_ƒEƒ“ƒ[ƒh‚Å‚«‚Ü‚·)
\footnote{http://www.math.tohoku.ac.jp/~kuroki/LaTeX/OpenCampus2009-Problem.pdf}.
“d‘ì‚ð‚œ‚œ‚¯‚Ί撣‚ê‚Ü‚·‚ª‚±‚ê‚̓iƒ“ƒZƒ“ƒX‚Å‚·.
‚à‚¿‚ë‚ño‘èŽÒ‚à‚»‚Ì‚æ‚€‚É‰ð‚­‚±‚Æ‚ðˆÓ}‚µ‚Ä‚¢‚Ü‚¹‚ñ.
–â‘è‚ð‰ð‚­‚œ‚ß‚É‚¢‚­‚‚©ƒqƒ“ƒg‚ª—^‚Š‚ç‚ê‚Ä‚š‚è,ƒqƒ“ƒg‚ð—p‚¢‚é‚Æ‚¿‚á‚ñ‚Æ‰ð‚¯‚é‚æ‚€‚É‚È‚Á‚Ä‚¢‚Ü‚·.
‚µ‚©‚µ, ƒqƒ“ƒg‚ª³‚µ‚¢‚±‚Ƃ̏ؖŸ‚ª–â‘蕶’†‚Å‚Í—^‚Š‚ç‚ê‚Ä‚¢‚Ü‚¹‚ñ.
ŽÀ‚Í‚±‚̃qƒ“ƒg‚ɉB‚ê‚Ä‚¢‚éƒJƒ‰ƒNƒŠ‚ª‚Æ‚Ä‚à–Ê”’‚­, Ø–Ÿ‚ɂ͐®”˜_‚É‚š‚¯‚é—L–Œ‚Ȓ藝‚ð‚‚©‚¢‚Ü‚·.
‚±‚̃m[ƒg‚̓qƒ“ƒg‚Æ‚µ‚ď‘‚©‚ê‚Ä‚¢‚鎖ŽÀ‚ɑ΂µ‚ďؖŸ‚ð—^‚Š‚邱‚Æ‚ð–Ú“I‚ɏ‘‚¢‚Ä‚ ‚è‚Ü‚·.
‘㐔Šw‚É‚š‚¯‚éŒQ˜_‚̏‰•à‚Ɛ®”˜_‚̍‡“¯Ž®‚̐«Ž¿‚Ì’mŽ¯‚ª‚ ‚ê‚Γǂ߂é‚æ‚€‚É‚È‚Á‚Ä‚¢‚Ü‚·.
\begin{center}
\shadowbox{zŠÂ¬”}
\end{center}

ˆÈ‰º‚̏¬”‚́u142857v‚Æ‚¢‚€”‚Ì—ñ‚ªŽüŠú“I‚É‚ž‚Á‚Æ‚È‚ç‚Ô‚à‚Ì‚É‚È‚Á‚Ä‚¢‚Ü‚·.
‚±‚Ì‚æ‚€‚ȏ¬”‚Í\underline{zŠÂ¬”}‚ƌĂ΂ê‚Ä‚¢‚Ü‚·.
\[
0.142857142857142857\cdots
\]
u142857v‚Æ‚¢‚€—ñ‚ð‚±‚Ì¬”‚Ì\textbf{zŠÂß}‚ƌĂԂ±‚Æ‚É‚µ‚Ü‚·.‚±‚ê‚Í $1/7$ ‚Ì’l‚Å‚·‚ª,
zŠÂß‚Ÿ‚¯‚ð—p‚¢‚Ä
\[
\frac{1}{7}=0.\dot{1}4285\dot{7}
\]
‚Æ•\‹L‚·‚é•û–@‚à‚ ‚è‚Ü‚µ‚œ‚Ë.
zŠÂß‚̐”‚Ì’·‚³‚ð\textbf{zŠÂß‚Ì’·‚³}‚Æ‚æ‚Ñ‚Ü‚µ‚å‚€.
—á‚Š‚Î, $1/7$ ‚̏ꍇ‚͏zŠÂß‚Ì’·‚³‚Í $6$ ‚Å‚·.
ˆÈ‰º, •ª•ê‚É—ˆ‚鐔‚Í $2,5$ ˆÈŠO‚Ì‘f”‚ÉŒÀ’肵‚Ü‚·.
–â‘è‚Æ‚µ‚Ä‹“‚°‚œ $12377$ ‚àŒvŽZ‚·‚ê‚Αf”‚ɂȂ邱‚Æ‚ª‚í‚©‚è‚Ü‚·.
ˆê‚–œ‘è‚ð‹“‚°‚Ä‚š‚«‚Ü‚·.
\begin{proposition*}
$p$ ‚ð $2,5$ ˆÈŠO‚Ì‘f”‚Æ‚·‚é.
‚±‚Ì‚Æ‚« $1/p$ ‚͏zŠÂ¬”‚É‚È‚é.
\end{proposition*}
\begin{proof}
$p$ ‚Í $2,5$ ˆÈŠO‚Ì‘f”‚È‚Ì‚Å, $10$ ‚ƌ݂¢‚É‘f‚ɂȂ邱‚Æ‚É’ˆÓ‚·‚é.
$10$ ‚Ì $\F_p=\Z/p\Z$ ‚̏æ–@ŒQ $\F_p^\times$ ‚É‚š‚¯‚éˆÊ”‚ð $e$ ‚Æ‚š‚­.‚‚܂è, $e$ ‚ð
\[
10^e\equiv 1\mod\ p.
\]
‚ð‚Ý‚œ‚·Å¬‚̐³‚̐”‚Ì‚±‚Æ‚Æ‚µ‚Ü‚·.
‚±‚Ì‚Æ‚« $10^e-1=pa$ ‚Æ‚¢‚€³‚̐®” $a$ ‚ª‘¶Ý‚·‚é‚Ì‚Å
\[
\frac{1}{p}=\frac{a}{pa}=\frac{a}{10^e -1}=\frac{a}{10^e}\frac{1}{1-\left(\frac{1}{10^e}\right)}
=\frac{a}{10^e}+\frac{a}{10^{2e}}+\frac{a}{10^{3e}}+\cdot \cdot \cdot .
\]
 $a<ap<10^e$ ‚©‚炱‚ê‚Í $1/p$ ‚ªzŠÂ¬”‚ɂȂ邱‚Æ‚ðˆÓ–¡‚·‚é.—á‚Š‚Î $a=123,10^e=1000$ ‚ðl‚Š‚é‚Æ‘æˆê€‚Í $0.123$ ‚Å‘æ“ñ€–Ú‚Í $0.000123$ ‚Æ‚È‚é.
‘æŽO€–Ú‚à“¯—l.
\end{proof}

ã‚ŏq‚ׂœ–œ‘è‚©‚ç $2$, $5$ ˆÈŠO‚Ì‘f”‚Ì‹t”‚Í•K‚žzŠÂ¬”‚ɂȂ邱‚Æ‚ª‚í‚©‚è‚Ü‚µ‚œ.
‘O–œ‘è‚̏ؖŸ‚ð‚æ‚­‚Ý‚é‚Æ $e$ ‚͏zŠÂß‚Ì’·‚³‚É‚È‚Á‚Ä‚¢‚邱‚Æ‚ª‚í‚©‚è‚Ü‚·.
—LŒÀŒQ‚̈ê”ʘ_‚É‚æ‚Á‚Ä $e$ ‚Í $p-1$ ‚Ì–ñ”‚ɂȂ邱‚Æ‚ª‚í‚©‚è‚Ü‚·.
\begin{center}
\shadowbox{ƒqƒ“ƒg(1)}
\end{center}
@ƒI[ƒvƒ“ƒLƒƒƒ“ƒpƒX‚̃qƒ“ƒg (1) ‚ðÐ‰î‚µ‚Ü‚·. ‚±‚Ì‚±‚Æ‚Í’Œ‘O‚ɏq‚ׂœ‚±‚Æ‚©‚ç–Ÿ‚ç‚©‚Å‚·.
\begin{itembox}[|]{ƒqƒ“ƒg(1)}
$2$, $5$ ˆÈŠO‚Ì‘f” $p$ ‚ɂ‚¢‚Ä $1/p$ ‚ðzŠÂ¬”‚Å•\ŽŠ‚µ‚œ‚Æ‚«, zŠÂß‚Ì’·‚³‚Í•K‚ž $p-1$ ‚Ì–ñ”‚É‚È‚é.
\end{itembox}
@‚Z‚Ì‹³‰È‘”I‚ð“ǂނƁu—L—”‚͏¬”‚É’Œ‚·‚Æ—LŒÀ¬”‚©zŠÂ¬”‚ɂȂ邱‚Æ‚ª’m‚ç‚ê‚Ä‚¢‚év‚Æ‚¢‚€‹Lq‚ª‚ ‚è‚Ü‚·‚ª, ã‚Ì–œ‘è‚Í“ÁŽê‚ȏꍇ‚ÌŽå’£‚É‚È‚Á‚Ä‚¢‚Ü‚·.
—]—Í‚ª‚ ‚ê‚Έê”ʂ̏ꍇ‚ɑ΂µ‚Ă̏ؖŸ‚ðl‚Š‚Ä‚Ý‚Ä‚­‚Ÿ‚³‚¢.
‹t‚ɏzŠÂ¬”‚Í•ª”‚ÌŒ`‚É’Œ‚¹‚é‚Æ‚¢‚€‚̂͐”I\!I\!I‚̃eƒLƒXƒg‚ðŒ©‚é‚Æ‚í‚©‚è‚Ü‚·.
\begin{center}
\shadowbox{ƒqƒ“ƒg(2)}
\end{center}

‚³‚Ä, ‘Oƒy[ƒW‚É‚à‚ ‚è‚Ü‚µ‚œ‚æ‚€‚É
\[
\frac{1}{7}=0.\dot{1}4285\dot{7}
\]
‚ª¬‚è—§‚Á‚Ä‚¢‚Ü‚µ‚œ (“d‘ì‚ð‚œ‚œ‚©‚ž‚ÉŽèŒvŽZ‚ÅŠm‚©‚ß‚Ä‚­‚Ÿ‚³‚¢).
142857 ‚ªzŠÂß‚Å‚·‚ª,“Á‚É’·‚³‚ª‹ô”‚ÆŒŸ‚€‚±‚Æ‚É’ˆÓ‚µ‚Ä‚š‚«‚Ü‚µ‚å‚€.
‘O”Œ‚̐”—ñu142v‚𕁒ʂ̐”u‚ЂႭ‚æ‚ñ‚¶‚ã‚€‚Ɂv,Œã”Œ‚̐”—ñu857v‚𕁒ʂ̐”u‚Í‚Á‚Ò‚á‚­‚²‚¶‚ã‚€‚Ȃȁv
‚Æ‚š‚à‚¢‚Ü‚µ‚å‚€.
ŠÈ’P‚ÈŒvŽZ‚Å
\[
142+857=999
\]
‚Æ‚È‚è‚Ü‚·.
‚ ‚è‚á‚܁[•sŽv‹c! (‚ÆŽv‚Á‚Ä‚­‚Ÿ‚³‚¢).
“d‘ì‚ð‚œ‚œ‚¢‚Ä‚à‚¢‚¢‚Ì‚Å $p=11,13$ ‚ȂǂŐ¬—§‚·‚é‚©‚Ç‚€‚©‚à‚œ‚ß‚µ‚Ä‚­‚Ÿ‚³‚¢:
\begin{example}
\begin{align*}
1/11&=0.\dot{0}\dot{9}\ ,
\\1/13&=0.\dot{0}7692\dot{3}\ ,
\\1/17&=0.\dot{0}58823529411764\dot{7}\ ,
\\1/19&=0.\dot{0}5263157894736842\dot{1}\ .
\end{align*}
\end{example}
ƒqƒ“ƒg $(2)$ ‚͈ȉº‚Ì’Ê‚è‚Å‚·.
%%%%%%%%%%%%%%%%%%%%%%%%%%%%%%%%%%%%%%%%%%%%%%%%%%%%%%%%%%%%%%%%%%%%%%%%%%
\begin{itembox}[|]{ƒqƒ“ƒg(2)}
 $p$ ‚ð $2,5$ ˆÈŠO‚Ì $1/p$ ‚̏zŠÂß‚Ì’·‚³‚ª‹ô”‚Ì‚Æ‚«,zŠÂß‚ð‘O”Œ,Œã”Œ‚É•ª‚¯‚Ä‚»‚ê‚Œ‚ê‚ð $a,b$ ‚Ə‘‚­‚Æ $a+b=99\cdots 9$ ‚ª¬—§‚·‚é.
\end{itembox}
%%%%%%%%%%%%%%%%%%%%%%%%%%%%%%%%%%%%%%%%%%%%%%%%%%%%%%%%%%%%%%%%%%%%%%%%%%
\begin{proof}[\underline{$\bullet$ ƒqƒ“ƒg $\mathrm{(2)}$ ‚̏ؖŸ}]
 $e$ ‚ð $\F_p^\times$ ‚É‚š‚¯‚é $10$ ‚̈ʐ”‚Æ‚·‚é. $e$ ‚Í‹ô”‚È‚Ì‚Å $e=2e'$ ‚Æ’u‚­‚Æ, $10^{e'}$ ‚Í
\[
x^2\equiv 1 \mod p
\]
‚̉ð‚Æ‚¢‚€‚±‚Æ‚©‚ç
 $10^{e'} \equiv -1 \mod p$ ‚ª¬—§‚·‚é.
‚æ‚Á‚Ä
\[
\frac{10^{e'}+1}{p}
\]
‚͐®”‚Æ‚È‚é. $1/p$ ‚ð­”‚Å‚ ‚ç‚킵‚Ä
\begin{align*}
\frac{1}{p}=0.&\dot{a_1}a_2\cdots a_{e'}a_{e'+1}\cdots a_{e-1}\dot{a_{e}}
\intertext{‚Æ‚·‚é. ‚±‚Ì‚Æ‚«}
\frac{10^{e'}}{p}=a_1a_2\cdots a_{e'}.&a_{e'+1}\cdots a_{e-1}a_{e}\cdots
\end{align*}
‚Æ‚È‚é‚©‚ç $1=0.999\cdots $ ‚Æ‚¢‚€Ž–ŽÀ‚ð”F‚ß‚ê‚Î $a_n+a_{e'+n}=9\ (n=1,2,\ldots ,e')$ ‚ª‚µ‚œ‚ª‚€.
\end{proof}
%%%%%%%%%%%%%%%%%%%%%%%%%%%%%%%%%%%%%%%%%%%%%%%%%%%%%%
\begin{remark*}
’l‚ª­‚È‚¢‘f”‚Ÿ‚¯‚Å‚Ý‚é‚Æ $2$, $5$ ˆÈŠO‚Ì‘S‚Ä‚Ì‘f”‚Ì‹t”‚̏zŠÂß‚Í‹ô” ! ‚ÆŽv‚¢‚œ‚­‚È‚è‚Ü‚·‚ª,
ˆê”ʂɐ³‚µ‚­‚Í‚È‚¢‚Å‚·. ŠÈ’P‚È”œ—á‚Æ‚µ‚Ä $p=3$ ‚â $p=37$ ‚̏ꍇ‚Å‚·.
ŽÀÛ,$1/3=0.\dot{3}$ $1/37=0.\dot{0}2\dot{7}$ ‚Å‚·. ‚·‚±‚µ’·‚­‚È‚é—á‚Æ‚µ‚Ä $p=31$ ‚ª‚ ‚è‚Ü‚·. ŽÀÛ. $1/p=31=0.\dot{0}3225806451612\dot{9}$ ‚Æ‚È‚Á‚ďzŠÂß‚Ì’·‚³‚Í $15$ ‚Å‚·.
\end{remark*}
%%%%%%%%%%%%%%%%%%%%%%%%%%%%%%%%%%%%%%%%%%%%%%%%%%%%%%%
ƒqƒ“ƒg $(2)$ ‚É‘±‚¢‚ăqƒ“ƒg $(2')$ ‚ª—pˆÓ‚³‚ê‚Ä‚¢‚Ü‚·.
\begin{itembox}[|]{ƒqƒ“ƒg$(2')$}
 $p$ ‚ð $2,5$ ‚Å‚È‚¢‘f”‚Æ‚·‚é.
$1/p$ ‚ð¬”‚Å•\‚µ,
¬”“_‘æ $(p-1)$ Œ…‚Ü‚Å‚ð‘O”Œ $a$ ‚Æ Œã”Œ $b$ ‚ɂ킯‚œ‚Æ‚·‚é.
‚±‚Ì‚Æ‚«, Œã”Œ‚̍ŏ‰‚ÌŒ…‚ª $9$ ‚È‚ç‚΃qƒ“ƒg $(2)$ ‚Æ“¯—l‚É $a+b=99\cdots 9$ ‚ª¬—§‚·‚é.
\end{itembox}

u$((p-1)/2+1)$ Œ…–Ú‚ª $9$v‚Æ‚È‚é‚Æ‚«‚Í‚Ç‚ñ‚ÈŽžH‚Æ‚¢‚€‹^–₪o‚Ü‚·‚ª, ‚»‚Ì‚±‚œ‚Š‚̓qƒ“ƒg(3) ‚ðŒ©‚ê‚΂킩‚è‚Ü‚·.
ƒqƒ“ƒg $(2')$ ‚̏ؖŸ‚ÍŒã‰ñ‚µ‚É‚µ‚¬‚É‚·‚·‚Þ‚±‚Æ‚É‚µ‚Ü‚·.
\begin{center}
\shadowbox{ƒqƒ“ƒg(3)}
\end{center}

ŽŸ‚̃qƒ“ƒg‚Í‘f”‚̏î•ñ‚©‚ç\underline{“Á’è‚ÌŒ…‚Ì’l‚ªŒˆ’è‚Å‚«‚é}‚±‚Æ‚ðq‚ׂĂ¢‚Ü‚·.
\begin{itembox}[|]{ƒqƒ“ƒg(3)}
\underline{$p$ ‚ð $11$ ˆÈã‚Ì‘f”‚Æ‚·‚é.} $1/p$‚̏¬”‘æ $((p-1)/2+1)$ Œ…–Ú‚Í $0$ ‚Ü‚œ‚Í $9$ ‚É‚È‚é.
‚à‚Á‚ÆŒŸ‚€‚ÆŽŸ‚ª¬‚è—§‚Â:
\begin{align*}
&p\equiv 1,3,9,13,27,31,37,39\quad\mod 40 &\Longleftrightarrow   &\quad\text{$1/p$‚̏¬”‘æ $((p-1)/2+1)$ Œ…–Ú‚Í $0$ .}
\\
&p\equiv 7,11,17,19,21,23,29,33\mod 40 &\Longleftrightarrow   &\quad\text{$1/p$‚̏¬”‘æ $((p-1)/2+1)$ Œ…–Ú‚Í $9$ .}
\end{align*}
\end{itembox}

ƒqƒ“ƒg(3)‚ðØ–Ÿ‚·‚é‚œ‚ß‚É‚¢‚­‚‚©®”˜_‚Å—L–Œ‚Ȓ藝‚ðŽ‚¿‚Ÿ‚µ‚Ü‚µ‚å‚€.
%%%%%%%%%%%%%%%%%%%%%%%%%%%%%%%%%%%%%%%%%%%%
\begin{center}
\shadowbox{Legendre symbol}
\end{center}
%%%%%%%%%%%%%%%%%%%%%%%%%%%%%%%%%%%%%%%%%%%%
\begin{definition*}
 $p$ ‚ðŠï‘f” (‚·‚È‚í‚¿, 2 ˆÈŠO‚Ì‘f”) ‚Æ‚µ‚Ü‚·.
$a$ ‚ð $p$ ‚ƌ݂¢‚É‘f‚Ȑ®”‚Æ‚·‚é.Legendre symbol $ \left(\frac{a}{p}\right)$ ‚ðˆÈ‰º‚Ì‚æ‚€‚É’è‹`‚µ‚Ü‚·:
\[
\left(\frac{a}{p}\right)
:=\begin{cases}
1 & x^2=a \mod p  \text{‚Æ‚¢‚€®” $x$ ‚ª‘¶Ý‚·‚é}\ ,
\\
-1 & x^2=a \mod p  \text{‚Æ‚¢‚€®” $x$ ‚ª‘¶Ý‚µ‚È‚¢}.
\end{cases}
\]
\end{definition*}
%%%%%%%%%%%%%%%%%%%%%%%%%%%%%%%%%%%%%%%%%%%%
\begin{itembox}[|]{Legendre symbol ‚̏”’藝}
%%%%%%%%%%%%%%%%%%%%%%%%%%%%%%%%%%%%%%%%%%%%
\begin{proposition*}
 $p$ ‚ðŠï‘f”‚Æ‚·‚é. $p$ ‚ƌ݂¢‚É‘f‚È$a,b\in \mathbb{Z}$ ‚ɂ‚¢‚ÄŽŸ‚ª¬‚è—§‚Â.
\begin{align*}
& a\equiv b \mod p \Longrightarrow \left(\frac{a}{p}\right)= \left(\frac{b}{p}\right)\ .
\\
& \left(\frac{ab}{p}\right)= \left(\frac{a}{p}\right) \left(\frac{b}{p}\right)\ .
\\
&\left(\frac{a}{p}\right)\equiv a^{\frac{p-1}{2}} \mod p\quad \text{(Euler ‚Ì‹K€)}.
\end{align*}
\end{proposition*}
\begin{theorem*}
$p,\ q$ ‚ðˆÙ‚È‚éŠï‘f”( 2 ˆÈŠO‚Ì‘f”) ‚Æ‚·‚邱‚ÌŽž, ŽŸ‚ª‚È‚è‚œ‚Â.
\begin{align*}
(•œ•ûè—]‚Ì‘ŠŒÝ–@‘¥)\ &\left(\frac{q}{p}\right)=\left(\frac{p}{q}\right)=(-1)^{\frac{p-1}{2}\frac{q-1}{2}}\ .
\\
(‘æˆê•â[–@‘¥)\ &\left(\frac{-1}{p}\right)=(-1)^{\frac{p-1}{2}}\
=
\begin{cases}
1  & p\equiv 1\mod 4\ ,
\\
-1 & p\equiv -1\mod 4\ .
\end{cases}
\\
(‘æ“ñ•â[–@‘¥)\ &\left(\frac{2}{p}\right)=(-1)^{\frac{p^2-1}{8}}
=
\begin{cases}
1  & p\equiv \pm 1\mod 8\ ,
\\
-1 & p\equiv \pm 3\mod 8\ .
\end{cases}
\end{align*}
\end{theorem*}
\end{itembox}

‚Å‚Í, ƒqƒ“ƒg (3) ‚ð‰ð‚­‚œ‚߂̏€”õ‚Æ‚µ‚Ä
\[
 \left(\frac{10}{p}\right)=1
\]
‚ª¬‚è—§‚•K—v\•ªðŒ‚ðŒ©‚Â‚¯‚Ü‚µ‚å‚€.
‚œ‚Ÿ‚µ, ‚±‚±‚Å‚Í $p$ ‚Í $5$ ˆÈŠO‚ÌŠï‘f”‚Æ‚µ‚Ü‚µ‚å‚€.
\begin{align*}
& \left(\frac{5}{p}\right)= \left(\frac{p}{5}\right)\equiv p^2 \!\mod 5
 =
 \begin{cases}
1  & p\equiv \pm 1\mod 5,
\\
-1 & p\equiv \pm 2\mod 5
 \end{cases}
\quad_,
&\left(\frac{2}{p}\right)=(-1)^{\frac{p^2-1}{8}}
=
\begin{cases}
1  & p\equiv \pm 1\mod 8,
\\
-1 & p\equiv \pm 3\mod 8.
\end{cases}
\end{align*}
]‚Á‚Ä,
\[
\left(\frac{10}{p}\right)= \left(\frac{2}{p}\right) \left(\frac{5}{p}\right)=1
\]
‚É’ˆÓ‚·‚é‚Æ
\[
\begin{cases}
p\equiv a \mod 5,
\\
p\equiv b \mod 8
\end{cases}
\]
‚Æ‚¢‚€‡“¯Ž®‚̘A—§•û’öŽ®‚ð–ž‚œ‚·‘f” $p$ ‚ð‹‚ß‚éŒvŽZ‚ª‚Ç‚€‚µ‚Ä‚à•K—v‚Å‚·.’†‘Ž®è—]’藝‚̈ê”ʘ_‚©‚ç
ã‚ÌðŒ‚ð–ž‚œ‚· $p$ ‚Í $40$ ‚ð–@‚Æ‚µ‚Ä‚³‚Ÿ‚Ü‚è,
‹ï‘Ì“I‚É‚Í $-24a+25b \mod 40$ ‚ª‹‚ß‚œ‚¢’l‚É‚È‚è‚Ü‚·.
‚¿‚Ü‚¿‚ÜŒvŽZ‚µ‚Ä‚¢‚¯‚ÎŽŸ‚ÌŒ‹‰Ê‚ð“Ÿ‚é‚Í‚ž‚Å‚·.
\begin{itembox}[|]{ŒvŽZŒ‹‰Ê}
$p$‚ð $2,5$ ˆÈŠO‚Ì‘f”‚Æ‚·‚é.
\begin{align*}
p&\equiv 1,3,9,13,27,31,37,39\,&\mod 40 &\quad\Leftrightarrow &\left ( \frac{10}{p} \right )&=1&\Leftrightarrow  &
\quad 10^{\frac{p-1}{2}}\equiv 1 \phantom{-}\mod p.
\\
p&\equiv 7,11,17,19,21,23,29,33\,&\mod 40&\quad\Leftrightarrow &\left ( \frac{10}{p} \right )&=-1&\Leftrightarrow  &
\quad 10^{\frac{p-1}{2}}\equiv -1 \mod p.
\end{align*}
‚ª¬—§‚·‚é.
\end{itembox}
\begin{proof}[\underline{$\bullet$ ƒqƒ“ƒg $\mathrm{(3)}$ ‚̏ؖŸ}] $10^{(p-1)/2}$‚ð $p$ ‚ÅŠ„‚Á‚œ—]‚è $10^{(p-1)/2}\mod p$ ‚ðŒvŽZ‚·‚é.
‚±‚ê‚Í $x$ ‚ÉŠÖ‚·‚é•û’öŽ® $x^2\equiv 1 \mod p$ ‚̉ð‚È‚Ì‚Å $10^{(p-1)/2}\mod p\equiv \pm 1$ ‚ª‚í‚©‚é.
\begin{itemize}
\item \underline{$10^{\frac{p-1}{2}}\equiv 1 \mod p$ ‚̏ꍇ.}
\\
@zŠÂß‚Ì’·‚³‚Í $(p-1)/2$ ‚Ì–ñ”‚É‚È‚é.
¬”“_‘æ $(p-1)/2$ Œ…–Ú‚Ì’l‚ð $\oplus$ ‚Æ’u‚­. $1/p$ ‚ð¬”‚Å•\ŽŠ‚µ‚œ‚Æ‚«
\[
\frac{1}{p}=0.\underset{zŠÂß}{\underbrace{0\cdot\,\cdot\ \cdot\,}}\,\underset{zŠÂß}{\underbrace{0\cdot\,\cdot\ \cdot\,}}\ \cdots\ \underset{zŠÂß}{\underbrace{0\cdot\,\cdot\ \cdot\oplus}}\,\underset{zŠÂß}{\underbrace{0\cdot\,\cdot\ \cdot\,}}\cdots
\]
‚Æ‚È‚Á‚Ä‚¢‚邱‚Æ‚ðˆÓ–¡‚·‚é
( $p$ ‚ª $11$ ˆÈã‚Ì‘f”‚Ȃ̂ŏ¬”“_ $1$ Œ…–Ú‚ª $0$ ‚É‚È‚Á‚Ä‚¢‚邱‚Æ‚É’ˆÓ‚µ‚œ‚¢).
‚æ‚Á‚﬐”“_‘æ $(p-1)/2+1$ Œ…–Ú‚Ì’l‚Í $0$ ‚Æ‚È‚é.
\\
\item \underline{$10^{\frac{p-1}{2}}\equiv -1 \mod p$ ‚̏ꍇ.} ŽŸ‚̍‡“¯Ž®‚ª¬‚è—§‚‚±‚Æ‚É’ˆÓ‚·‚é:
\[
10^{\frac{(p-1)}{2}}\equiv -1 \equiv p-1 \mod p ,\quad 10^{\frac{(p-1)}{2}+1}\equiv -10 \equiv p-10 \mod p.
\]
ˆê•û, $p-10 > 0$ ‚Æ‚¢‚€‚±‚Æ‚Æ, ‚í‚ê‚í‚ꂪ ­”‘æ $\frac{p-1}{2}+1$ Œ…–Ú‚ð•MŽZ‚·‚é‚Æ‚«‚É
\[
(p-1)10 -p a_{\frac{p-1}{2}+1}=p-10
\]
‚Æ‚¢‚€ŒvŽZ‚ðŽÀs‚µ‚Ä‚¢‚邱‚Æ‚É’ˆÓ‚·‚é.
‚±‚ê‚©‚ç $a_{\frac{p-1}{2}+1}=9.$ ‚ª‚í‚©‚é.
\\
Œã‚͏ã‚ÌŒvŽZŒ‹‰Ê‚ƍ‡‚킹‚é‚ÆŽå’£‚ª“Ÿ‚ç‚ê‚é.
\end{itemize}
\end{proof}
\newpage
ã‚̏ؖŸ‚©‚玟‚Ì‚±‚Æ‚ª‚·‚®•ª‚©‚è‚Ü‚·.
‚±‚ê‚̓qƒ“ƒg(3')‚Æ‚µ‚Ä—^‚Š‚ç‚ê‚Ä‚¢‚Ü‚·.
%%%%%%%%%%%%%%%%%%%%%%%%%%%%%%%%%%%%%%%%%%%%%%%%%%%%%%%%%%%
\begin{itembox}[|]{ƒqƒ“ƒg(3')}
$p$ ‚ð $2,5$ ˆÈŠO‚Ì‘f”‚Æ‚·‚é.
\begin{align*}
p&\equiv 1,3,9,13,27,31,37,39\, \quad\mod 40    \Longleftrightarrow  \text{ $1/p$ ‚̏zŠÂß‚Ì’·‚³‚Í $(p-1)/2$ ‚Ì–ñ”.}
\\
p&\equiv 7,11,17,19,21,23,29,33\,\mod 40 \Longleftrightarrow \text{ $1/p$ ‚̏zŠÂß‚Ì’·‚³‚Í $(p-1)/2$ ‚Ì–ñ”‚Å‚È‚¢.}
\end{align*}
\end{itembox}
%%%%%%%%%%%%%%%%%%%%%%%%%%%%%%%%%%%%%%%%%%%%%%%%%%%%%%%%%%%

ÅŒã‚É, ƒqƒ“ƒg(2')‚̏ؖŸ‚ð—^‚Š‚Ü‚µ‚å‚€.
\begin{center}
\shadowbox{ƒqƒ“ƒg(2')‚̏ؖŸ}
\end{center}
\begin{proof}
$p$ ‚Í $11$ ˆÈã‚Ì‘f”‚Æ‚µ‚Ä‚æ‚¢. $1/p$ ‚̏¬”‘æ $(p-1)/2+1$ Œ…–Ú‚ª9‚È‚ç‚Î, ¡‚܂ł̍lŽ@‚É‚æ‚Á‚Ä
\[
10^{\frac{p-1}{2}}\equiv -1 \mod p
\]
‚Å‚È‚¢‚Æ‚¢‚¯‚È‚¢.
\[
\frac{1}{p}=0.a_1a_2\dots a_{\frac{p-1}{2}}a_{\frac{p-1}{2}+1}\cdots a_{p-1}\cdots
\]
‚Æ‚·‚é‚Æ, ƒqƒ“ƒg $(2)$ ‚̏ؖŸ‚É‚š‚¯‚é $e'$ ‚Ì•”•ª‚ð $\frac{p-1}{2}$ ‚É’u‚«Š·‚Š‚œŒ‹‰Ê‚É‚æ‚ê‚Î
\[
a_n+a_{\frac{p-1}{2}+n}=9\quad \left(n=1,2,\ldots ,\frac{p-1}{2}\right).
\]
\end{proof}
%%%%%%%%%%%%%%%%%%%%%%%%%%%%%%%%%%%%%%%%%%%%%%%
%%%%%%%%%%%%%%%%%%%%%%%%%%%%%%%%%%%%%%%%%%%%%%%
\begin{center}
\shadowbox{•â‘«}
\end{center}
%%%%%%%%%%%%%%%%%%%%%%%%%%%%%%%%%%%%%%%%%%%%%%%
%%%%%%%%%%%%%%%%%%%%%%%%%%%%%%%%%%%%%%%%%%%%%%%

ŽÀ‚Í \underline{$11$ ˆÈã‚Ì‘f”‚ŏ¬”‘æ $((p-1)/2+1)$ Œ…–Ú‚ª $9$ ‚È‚ç‚Î, zŠÂß‚Ì’·‚³‚Í‹ô”‚Æ‚È‚é}‚±‚Æ‚ª‚í‚©‚è‚Ü‚·.
\begin{proof}
ƒqƒ“ƒg $(3')$ ‚É‚æ‚Á‚Ä $1/p$ ‚̏zŠÂß‚Ì’·‚³‚ð $e$ ‚Æ‚µ‚œ‚Æ‚«, $e$ ‚Í $p-1$ ‚Ì–ñ”‚Å‚©‚ $(p-1)/2$‚Ì–ñ”‚Å‚È‚¢‚Æ‚¢‚€‚±‚Æ‚ª‚í‚©‚è‚Ü‚·.
\[
a:=\frac{p-1}{2}=2^{k_0} q_1^{k_1}\dots q_r^{k_r}
\]
‚Æ‘fˆö”•ª‰ð‚·‚é.
‚œ‚Ÿ‚µ, $q_1,\dots,q_r$ ‚ÍŠï‘f”‚Å, $k_0 \geq 0 , k_1,\dots , k_r \geq 1$ ‚Å‚ ‚é.@
‚±‚Ì‚Æ‚«
\[
p-1=2^{k_0+1}q_1^{k_1}\dots q_r^{k_r}
\]
‚Æ‚È‚é. $e$ ‚Í $p-1$ ‚Ì–ñ”‚È‚Ì‚Å $e$ ‚Í
\[
e=2^{f_0}q_1^{f_1}\dots q_r^{f_r}
\]
‚Ə‘‚¯‚é.
‚±‚ê‚©‚ç
\[
\begin{cases}
 f_0\leq k_0+1 				 &\cdots (1)
\\
\\
 f_i\leq k_i\quad (1\leq i\leq r)&\cdots (2)
\end{cases}
\]
‚ª¬‚è—§‚‚±‚Æ‚É’ˆÓ‚µ‚æ‚€.
ˆê•û‚Å,
$e$ ‚Í $a$ ‚Ì–ñ”‚Å‚È‚¢‚Ì‚Å $f_j>k_j$ ‚Æ‚È‚é $j\in \{\, 0,1,\dots ,r\}$ ‚ª‘¶Ý‚µ‚È‚¢‚Æ‚¢‚¯‚È‚¢‚ª,
ã‚Ì $\cdots(2)$ ‚©‚ç $j=0$ ‚Å‚È‚¢‚Æ‚¢‚¯‚È‚¢‚±‚Æ‚ª•ª‚©‚é.
‚±‚ê‚Í $0\leq k_0<f_0$, ŒÌ‚É, $1\leq f_0$ ‚ðˆÓ–¡‚·‚é‚Ì‚Å, $e$ ‚ª‹ô”‚Å‚ ‚邱‚Æ‚ª•ª‚©‚é.
\end{proof}
%%%%%%%%%%%%%%%%%%%%%%%%%%%%%%%%%%%
\begin{remark*}

$11$ ˆÈã‚Ì‘f”‚ŏ¬”‘æ $((p-1)/2+1)$ Œ…–Ú‚ª $9$ ‚Å‚È‚¢ (‚‚܂è 0 ‚Ì‚Å‚ ‚é) ‚Æ‚«‚Å‚à
zŠÂß‚Ì’·‚³‚Í‹ô”‚ɂȂ鎞‚Í‚ ‚è‚Ü‚·.
—á‚Š‚Î $p=13$, 89, 157 ‚̏ꍇ‚Å‚·( 40 ‚ÅŠ„‚Á‚œ—]‚è‚ðl‚Šƒqƒ“ƒg (3) ‚ƏƂ炵‡‚킹‚Ä‚Ý‚Ü‚µ‚å‚€).
‚±‚ê‚ç‚̏zŠÂß‚Ì’·‚³‚Í‚»‚ê‚Œ‚ê 6, 44, 78 ‚É‚È‚è‚Ü‚·.
\end{remark*}
%%%%%%%%%%%%%%%%%%%%%%%%%%%%%%%%%%%%%%%%
\newpage
%%%%%%%%%%%%%%%%%%%%%%%%%%%%%%%%%%%%%%%%
%%%%%%%%%%%%%%%%%%%%%%%%%%%%%%%%%%%%%%%%
\begin{center}
\shadowbox{Appendix 1 : ƒpƒ\ƒRƒ“‚ÅŠm‚©‚ß‚Ä‚Ý‚æ‚€}
\end{center}
%%%%%%%%%%%%%%%%%%%%%%%%%%%%%%%%%%%%%%%%

$p$ ‚ª‘å‚«‚­‚Ȃ邃ʏí‚Ì“d‘ì‚ł͑Ήž‚Å‚«‚Ü‚¹‚ñ. C++‚Å $1/p$ ‚ð­”‚É‚·‚éƒvƒƒOƒ‰ƒ€\footnote{ƒvƒƒOƒ‰ƒ€‚ÉŠÖ‚µ‚Ä‚Í‚ ‚Ü‚èÚ‚µ‚­‚È‚¢‚Ì‚Å–³‘Ê‚È•\‹L‚ª‚ ‚Á‚œ‚è‚·‚é‚©‚à‚µ‚ê‚Ü‚¹‚ñ
.}‚ðì‚è‚Ü‚µ‚œ.
\begin{verbatim}
#include<iostream.h>
int quotient(int,int);//a‚ð‚‚‚ÅŠ„‚Á‚œ‚Æ‚«‚̏€‚ð‚à‚Æ‚ß‚éB
int main(void)
{
int p ;
cout<<"‚±‚ê‚Í‚P/p‚ð¬”“WŠJ‚·‚éƒvƒƒOƒ‰ƒ€‚Å‚·.Š„‚鐔p‚ð“ü—Í‚µ‚Ä‚­‚Ÿ‚³‚¢."<<endl;
cout<<"p"<<endl;
cin>>p ;//Š„‚鐔‚ðp‚Æ‚š‚­B
cout<<"­”‘扜Œ…‚Ü‚Å‚ðo‚µ‚œ‚¢‚Å‚·‚©H‚œ‚Ÿ‚µA‚P‚O–œŒ…‚Ü‚Å‚ªŒÀŠE‚Å‚·."<<endl;
int m ;//o—Í‚·‚錅”‚ðm‚Æ‚š‚­B
cin>> m ;
int Q[100000] ;//o—Í‚µ‚œ’l‚ð“ü‚ê‚é” 
int q=quotient(10,p);
int r=10%p ;
for(int i=0;i<m+1;i++)
{
Q[i]=q;
q=quotient(10*r,p);
r=(10*r)%p;
}//‰äXlŠÔ‚ª•MŽZ‚ð‚·‚éŽè‘±‚«‚ðÄŒ»‚µ‚Ä‚¢‚éBa%b‚Åa‚ðb‚ÅŠ„‚Á‚œ—]‚è‚ð•\‚·B
cout<<"Œ‹‰Ê‚Í"<<endl;
cout<<"1/p=0.";
for(int i=0 ; i<m ; i++)
	{cout<<Q[i] ;}//ÅI“I‚ÈŒ‹‰Ê‚ðo—Í‚·‚é
cout<<""<<endl;
cout<<"‚¿‚È‚Ý‚É‘æ"<<m<<"Œ…–Ú‚Í"<<Q[m-1]<<"‚Å‚ ‚é."<<endl;//Œ…”‚ðo‚·
return 0;
}

int quotient(int a,int b)
{int s;
s=(a-(a%b))/b ;
return s ;
}
\end{verbatim}
%%%%%%%%%%%%%%%%%%%%%%%%%%%%%%%%%%%%%%%%%%%%%%%
\newpage
%%%%%%%%%%%%%%%%%%%%%%%%%%%%%%%%%%%%%%%%%%%%%%%
\begin{center}
\shadowbox{Appendix 2 : Legendre symbol ‚̏”’藝‚̏ؖŸ}
\end{center}
%%%%%%%%%%%%%%%%%%%%%%%%%%%%%%%%%%%%%%%%%%%%%%%

‚±‚±‚Å‚ÍLegendre symbol ‚̏ؖŸ‚ð—^‚Š‚邱‚Æ‚É‚µ‚Ü‚·.
\begin{proposition*}[Euler ‹K€]
$p$ ‚ðŠï‘f”‚Æ‚·‚é. $p$ ‚ƌ݂¢‚É‘f‚È $a\in \Z$ ‚ɑ΂µ‚Ä
\[
\left(\frac{a}{p}\right)\equiv a^{\frac{p-1}{2}}\mod p.
\]
\end{proposition*}
\begin{proof}
$a^\frac{p-1}{2}\equiv \pm 1 \mod p$ ‚Å‚ ‚é‚©‚玟‚ðŽŠ‚¹‚Ώ\•ª:
\begin{claim*}
\[
\left(\frac{a}{p}\right)=1\Leftrightarrow a^{\frac{p-1}{2}}=1\mod p.
\]
\end{claim*}
ŽÀÛ, $x^2\equiv a\mod p$ ‚Æ‚¢‚€ $x\in \Z$ ‚ª‘¶Ý‚µ‚œ‚Æ‚·‚é. $x$ ‚Æ $p$ ‚݂͌¢‚É‘f‚È‚Ì‚Å,
ƒtƒFƒ‹ƒ}[‚̏¬’藝‚©‚ç
\[
a^{\frac{p-1}{2}}\equiv x^{p-1}\equiv 1\mod p.
\]
‹t‚É,
$a^{\frac{p-1}{2}}=1\mod p$ ‚Ì‚Æ‚«, $g$ ‚ð $\F_{p}^\times$ ‚ÌŒŽŽnª‚Æ‚µ, $a=g^r \mod p$ ‚Æ‚È‚Á‚Ä‚¢‚é‚Æ‚·‚é.
\[
 g^{r{\frac{p-1}{2}}}\equiv a^{\frac{p-1}{2}}\equiv 1\mod p.
\]
‚Æ‚È‚é.
‚æ‚Á‚Ä, $r(p-1)/2$ ‚Í $p-1$ ‚Ì”{”. ‚æ‚Á‚Ä $r/2$ ‚͐®”. ‚±‚ê‚Í $a$ ‚Ì $g$ ‚ÉŠÖ‚·‚éŽw” $r$ ‚ª‹ô”‚ÆŒŸ‚€‚±‚Æ‚ðˆÓ–¡‚µ‚Ä‚¢‚é.
‚ä‚Š‚É
\[
\left(\frac{a}{p}\right)=1.
\]
\end{proof}
‚µ‚œ‚ª‚Á‚Ä, Legendre symbol ‚ðŽŸ‚Ì‚æ‚€‚É’è‹`‚µ’Œ‚·‚±‚Æ‚à‰Â”\‚Å‚·:
\begin{definition*}
$p$ ‚ðŠï‘f”‚Æ‚µ‚Ä, $a$ ‚ð $p$ ‚ƌ݂¢‚É‘f‚È ®”‚Æ‚·‚é.‚±‚Ì‚Æ‚«,
\[
\left(\frac{a}{p}\right):=a^\frac{p-1}{2} \mod p
\]
‚Æ’è‚ß‚é.
\end{definition*}
\begin{proposition*}
$p$ ‚ðŠï‘f”‚Æ‚·‚é. $a,b$ ‚ð $p$ ‚ƌ݂¢‚É‘f‚Ȑ®”‚Æ‚·‚é. ‚±‚Ì‚Æ‚«,
\[
\left(\frac{ab}{p}\right)=\left(\frac{a}{p}\right)\left(\frac{b}{p}\right).
\]
\end{proposition*}
\begin{proof}
Euler‹K€‚©‚ç
\[
\mathrm{L.H.S}= (ab)^\frac{p-1}{2}\mod p= a^\frac{p-1}{2}b^\frac{p-1}{2}\mod p=\mathrm{R.H.S}\ .
\]
\end{proof}

ˆÈã‚ŁuLegendre symbol ‚̏”’藝v‚ŏq‚ׂœ‘O”Œ‚Ì–œ‘è‚̏ؖŸ‚ªI‚í‚è‚Ü‚µ‚œ.
ˆÈ‰º, •â[–@‘¥, ‚š‚æ‚Ñ‘ŠŒÝ–@‘¥‚ðŽOŠpŠÖ”‚ð—p‚¢‚œ Serre‚̐”˜_u‹`(Šâ”g‘“X) ‚É‚Ì‚Á‚Æ‚Á‚ÄŽŠ‚µ‚Ä‚¢‚«‚Ü‚·. Œã”Œ‚̏ؖŸ‚Ì‚œ‚ß‚ÉŽŸ‚ɏq‚ׂéuGauss‚Ì•â‘èv‚ðŽŠ‚µ‚Ü‚·.
‚»‚Ì‚œ‚ß‚É‚Ü‚·, ‹L†‚̏€”õ‚©‚ç‚Í‚¶‚ß‚Ä‚¢‚«‚Ü‚µ‚å‚€.
$p$ ‚ðŠï‘f”‚Æ‚µ‚Ü‚·.
\[
S:=\left\{ 1,2,\cdots,\frac{p-1}{2} \right\},\ -S:=\left\{ -1,-2,\cdots,-\frac{p-1}{2} \right\}
\]
‚Æ’u‚¢‚Ä, $\F_{p}^\times$ ‚ÌŠ®‘S‘ã•\Œn‚ð
$S\cup (-S)$ ‚łƂ邱‚Æ‚É‚µ‚Ü‚·.
$\F_{p}^\times =S\cup -S$ ‚Ì‚à‚Æ‚Å, $a\in \F_{p}^\times , s\in S$ ‚ɑ΂µ
$as\equiv e_{s}(a)s_{a}\mod p$ ‚Æ‚¢‚€ $e_{s}(a)\in \{\pm 1\}, s_{a}\in S$ ‚ªˆêˆÓ‚É’è‚܂邱‚Æ‚É’ˆÓ‚µ‚Ü‚µ‚å‚€.
\begin{lemma*}[Gauss‚Ì•â‘è]
Šï‘f” $p$ ‚ƌ݂¢‚É‘f‚È $a\in \Z$ ‚ɂ‚¢‚Ä
\[
\left(\frac{a}{p}\right)=\prod_{s\in S}e_{s}(a).
\]
\end{lemma*}
\begin{proof}
\begin{claim*}
$a\in \Z,$ $s,s'\in S$ $(s\neq s')$ ‚ɂ‚¢‚Ä
$s_{a}\neq s'_{a}.$
\end{claim*}
ŽÀÛ, $s_{a}= s'_{a}$ ‚È‚ç‚Î
\[
as \equiv \pm s_a=\pm s'_{a}\equiv \pm as' (•¡‡”CˆÓ)
\]
‚Æ‚È‚é. $s\neq s'$ ‚Ÿ‚©‚ç $s=-s'$ ‚ð—v¿‚·‚邪,
\[
s,s'\in S=\left\{1,2,\cdots, \frac{p-1}{2}\right\}
\]
‚ðl‚Š‚é‚Æ, ‚±‚ê‚Í‚ ‚è“Ÿ‚È‚¢‚Æ‚¢‚€‚±‚Æ‚ª‚í‚©‚é. ‚±‚ê‚Å\textbf{Claim.}‚̏ؖŸ‚ªI‚í‚Á‚œ.

ˆÈã‚Å
\[
S\ni s\longmapsto s_{a}\in S
\]
‚̑Ήž‚ª‘S’PŽË‚ƂȂ邱‚Æ‚ª‚í‚©‚Á‚œ.
‚æ‚Á‚Ä
\begin{align*}
a^{\frac{p-1}{2}}\left(\prod_{s\in S}s \right)=\prod_{s\in S}as=\prod_{s\in S}e_{s}(a)s_a
=\left(\prod_{s\in S} e_s(a) \right)\left(\prod_{s\in S}s_a \right)=\left(\prod_{s\in S} e_s(a) \right)\left(\prod_{s\in S}s \right).
\end{align*}
\[
\therefore \left(\frac{a}{p}\right)=a^{\frac{p-1}{2}}=\left(\prod_{s\in S} e_s(a) \right).
\]
\end{proof}
\begin{proposition*}[‘æ“ñ•â[–@‘¥]
$p$‚ðŠï‘f”‚Æ‚·‚邱‚Ì‚Æ‚«,‚‚¬‚ª‚È‚è‚œ‚Â:
\[
\left(\frac{2}{p}\right)
=(-1)^{\frac{p^2-1}{8}}
=
\begin{cases}
1   &  \mathrm{if}\ p\equiv \pm 1\mod 8,
\\
-1  &  \mathrm{if}\ p\equiv \pm 5\mod 8.
\end{cases}
\]
\end{proposition*}
\begin{proof}
$p=3$ ‚̏ꍇ‚Í‚·‚®‚É‚í‚©‚é‚Ì‚Å $p\geq 5$ ‚̏ꍇ‚ðØ–Ÿ‚·‚é.
$s\in S$ ‚ɑ΂µ‚Ä
\[
2s\leq \frac{p-1}{2} \equiv e_{s}(2)=1
\] ‚È‚Ì‚Å
\[
n(p)=\left|\left\{\, s\in \Z\, ;\, \frac{p-1}{2}<2s\leq p-1 \,\right\}\right|
\]
‚Æ‚š‚­.
Gauss‚Ì•â‘è‚©‚ç
\[
\left(\frac{2}{p}\right)=\prod_{s\in S}e_{s}(2)=\prod_{\frac{p-1}{2}<2s\leq p-1}(-1) =(-1)^{n(p)}
\]
‚ª]‚€. $n(p)$ ‚ª‚Ç‚€‚È‚é‚©‚Å $\left(\frac{2}{p}\right)$ ‚Ì’l‚ª‚í‚©‚é‚Ì‚Å, $n(p)$ ‚̐U‚é•‘‚¢‚𒲂ׂ邱‚Æ‚É‚·‚é.
‘f” $p$ ‚ª 8 ‚ð–@‚Æ‚µ‚Ä $1$, $5$, $-1$, $-5$ ‚Ì‚Ç‚ê‚©‚ƍ‡“¯‚É‚È‚é.
ŠeX‚̏ꍇ‚ɂ‚¢‚ÄŽŸ‚ª¬‚è—§‚‚±‚Æ‚É’ˆÓ‚µ‚æ‚€.
\begin{align*}
p\equiv 1\mod 8&\Longrightarrow p=8l+1\ (^\exists l\in\Z\,)\ \Longrightarrow p=4k+1\quad (k=2l),
\\
p\equiv 5\mod 8&\Longrightarrow p=8l+5\ (^\exists l\in\Z\,)\ \Longrightarrow p=4k+1\quad (k=2l+1),
\\
p\equiv -1\mod 8&\Longrightarrow p=8l-1\ (^\exists l\in\Z\,)\ \Longrightarrow p=4k+3\quad (k=2l-1),
\\
p\equiv -5\mod 8&\Longrightarrow p=8l-5\ (^\exists l\in\Z\,)\ \Longrightarrow p=4k+1\quad (k=2(l-1))
\end{align*}
%%%%%%%%%%%%%%%%%%%%%%%%%%%%
\newpage
%%%%%%%%%%%%%%%%%%%%%%%%%%%%
\begin{claim*}
5 ˆÈã‚Ì‘f” $p$ ‚ɑ΂µ‚Ä,
\begin{align*}
&\bullet\  p=4k+1\ (^\exists k \in \Z\,) \Longrightarrow n(p)=k,
\\
&\bullet\  p=4k+3\ (^\exists k \in \Z\,) \Longrightarrow n(p)=k+1.
\end{align*}
\end{claim*}
ŽÀÛ
\begin{itemize}
\item $p=4k+1$ $(^\exists k \in \Z\,)$ ‚̏ꍇ
\[
\frac{p-1}{2}<2s\leq p-1
\Longleftrightarrow
2k<2s\leq 4k
\Longleftrightarrow
k<s\leq 2k.
\qquad \therefore n(p)=k.
\]
\item $p=4k+3$ $(^\exists k \in \Z\,)$ ‚̏ꍇ
\[
\frac{p-1}{2}<2s\leq p-1
\Longleftrightarrow
2k+1<2s\leq 4k+2
\Longleftrightarrow
k+1\leq s\leq 2k+1.
\qquad
\therefore n(p)=k+1.
\]
\end{itemize}
‚æ‚Á‚Ä\textbf{Claim.}‚ªŽŠ‚³‚ê‚œ.
‚±‚Ì‚±‚Æ‚©‚ç
\[
\left(\frac{2}{p}\right)
=
\begin{cases}
1   &  \mathrm{if}\ p\equiv \pm 1\mod 8,
\\
-1  &  \mathrm{if}\ p\equiv \pm 5\mod 8.
\end{cases}
\]
‚ª‚í‚©‚é.‚±‚Ì•ªŠòðŒ‚Í $(-1)^{\frac{p^2-1}{8}}$ ‚ƈê’v‚·‚邱‚Æ‚Í—eˆÕ‚É‚í‚©‚é.
\end{proof}
\begin{proposition*}[‘æˆê•â[–@‘¥]
\[
\left(\frac{-1}{p}\right)
=(-1)^{\frac{p-1}{2}}
=
\begin{cases}
1   &  \mathrm{if}\ p\equiv  1\mod 4,
\\
-1  &  \mathrm{if}\ p\equiv -1\mod 4.
\end{cases}
\]
\end{proposition*}
\begin{proof}
’è‹`‚©‚ç
\[
e_s(-1)=-1\ \forall s\in S
\]
‚ª‚í‚©‚é‚Ì‚Å, Gauss‚Ì•â‘è‚©‚ç
\[
\left(\frac{-1}{p}\right) =\prod_{s\in S}(-1)=(-1)^{\frac{p-1}{2}}
\]
‚Æ‚È‚é. $(-1)^{\frac{p-1}{2}}$ ‚ªðŒŽ®‚ƈê’v‚·‚é‚Ì‚Í—eˆÕ‚É‚í‚©‚é.
\end{proof}
‚¢‚æ‚¢‚æ•œ•ûè—]‚Ì‘ŠŒÝ–@‘¥‚̏ؖŸ‚ð‚Ý‚Ä‚¢‚±‚€.ŽŸ‚ÌŽOŠp–@‚Ì•â‘肪 Key ‚É‚È‚é.
\begin{proposition*}[ŽOŠp–@‚Ì•â‘è]
$q$ ‚𐳂̊‚Æ‚·‚é. ‚±‚Ì‚Æ‚«ŽŸ‚̍P“™Ž®‚ª¬‚è—§‚Â:
\[
\frac{\sin (qx)}{\sin x}=(-4)^{\frac{q-1}{2}}\prod_{t=1}^{\frac{q-1}{2}}\left(\sin^2 x-\sin^2\left(\frac{2\pi t}{q}\right)\right).
\]
\end{proposition*}
ŽOŠp–@•â‘è‚̏ؖŸ‚Ì‚œ‚ß‚ÉŽŸ‚Ì•â‘è‚ðŽŠ‚·.
\begin{lemma*}
³‚̐®” $n$ ‚ɑ΂µ‚Ä,Å‚ŽŸŒW”‚ª$(-4)^n$ ‚̐®”ŒW” $n$ ŽŸ‘œ€Ž®‚Å, ŽŸ‚ÌðŒ‚ð–ž‚œ‚· $\Phi_n(X)\in\Z [X]$ ‚ª‘¶Ý‚·‚é:
\[
\frac{\sin (2n+1)x}{\sin x}=\Phi_n (\sin^2 x).
\]
\end{lemma*}
\begin{proof}
ƒIƒCƒ‰[‚ÌŒöŽ®‚É‚æ‚Á‚Ä
\begin{align*}
e^{i(2n+1)x}
&=
(\cos x+i\sin x)^{2n+1}=\cos((2n+1)x)+i\sin((2n+1)x)
\\
&=
\sum_{j=0}^{2n+1}\binom{2n+1}{j}(\cos x)^{2n+1-j} (i\sin x)^j
\intertext{$j$ ‚ª‹ô”‚©Šï”‚©‚ŏꍇ•ª‚¯‚·‚é. }
&=
\sum_{j=0}^{n}\binom{2n+1}{2j}(\cos x)^{2n+1-2j}(-1)^j\sin^{2j}x
+i
\sum_{j=0}^{n}\binom{2n+1}{2j+1}(\cos x)^{2n-2j}(-1)^j(\sin x)^{2j+1}
\end{align*}
‚Æ‚È‚é.—Œ•Ó‚Ì‹••”‚ð‚Æ‚Á‚Đ®—‚·‚é‚Æ
\[
\frac{\sin(2n+1)x}{\sin x}=(-1)^n\sum_{j=0}^n\binom{2n+1}{2j+1}(\sin^2 x-1)^{n-j}\sin^{2j} x
\]
‚Æ‚È‚é.
\begin{claim*}
\[
\Phi_n(X):=(-1)^n\sum_{j=0}^n\binom{2n+1}{2j+1}(X-1)^{n-j}X^j
\]
‚Æ‚š‚­‚Æ,‚±‚ꂪ‹‚ß‚œ‚¢ $n$ ŽŸ‘œ€Ž®‚Å‚ ‚é.
\end{claim*}
ŽÀÛ,
\[
\frac{\sin (2n+1)x}{\sin x}=\Phi (\sin^2 x)
\]
‚Í–Ÿ‚ç‚©‚È‚Ì‚Å,Å‚ŽŸŒW”‚𒲂ׂê‚΂悢.
\begin{align*}
\Phi(X)\ ‚̍ō‚ŽŸŒW”
&=
(-1)^n\sum_{j=0}^n\binom{2n+1}{2j+1}
\\
&=
(-1)^n\sum_{j=0}^n\left(\binom{2n}{2j}+\binom{2n}{2j+1}\right)
\\
&=
(-1)^n\sum_{k=0}^{2n}\binom{2n}{k}=(-1)^n2^{2n}=(-4)^n.
\end{align*}
‚±‚ê‚ÅŽŠ‚·‚ׂ«‚±‚Æ‚ªŽŠ‚³‚ê‚œ.
\\
\end{proof}
\begin{proof}[ˆÈ‰º, ŽOŠp–@‚Ì•â‘è‚̏ؖŸ‚ðŽŠ‚·] $\Phi_(X):=\Phi_{\frac{q-1}{2}}(X)$ ‚Æ’u‚«,
\[
X_t:=\sin^2\left(\frac{2\pi t}{q}\right)\ \mathrm{for}\ t=1,2,\cdots,\frac{q-1}{2}
\]
‚Æ’u‚­.‘O•â‘è‚©‚ç
\[
\Phi(X_t)=\frac{\sin\left(q\cdot\frac{2\pi t}{q}\right)}{\sin\left(\frac{2\pi t}{q}\right)}=0\ \mathrm{for}\ t=1,2,\cdots,\frac{q-1}{2}
\]
‚Æ‚¢‚€‚±‚Æ‚ª‚í‚©‚é‚Ì‚Å,
\[
\Phi(X)=(-4)^{\frac{q-1}{2}}\prod_{t=1}^{\frac{q-1}{2}}\left(X-\sin^2\left(\frac{2\pi t}{q}\right)\right).
\]
‚æ‚Á‚Ä,
\[
\frac{\sin (qx)}{\sin x}=\Phi(\sin^2 x)=(-4)^{\frac{q-1}{2}}\prod_{t=1}^{\frac{q-1}{2}}\left(\sin^2 x-\sin^2\left(\frac{2\pi t}{q}\right)\right).
\]
\end{proof}

ÅŒã‚É‚í‚ê‚í‚ê‚̍ŏI–Ú•W‚Å‚ ‚é•œ•ûè—]‚Ì‘ŠŒÝ•ú‘—‚̏ؖŸ‚É“ü‚é.
ŽOŠpŠÖ”ŽüŠú«‚Æ $\sin (-x)=-\sin x$ ‚Æ‚¢‚€«Ž¿‚ð‚€‚Ü‚­—˜—p‚·‚邱‚Æ‚É‚æ‚Á‚Ä‚É‚æ‚Á‚Ä, Legendre symbol
\[
\left(\frac{p}{q}\right)
\]
‚ª $p$ ‚Æ $q$ ‚ð“ü‚ꊷ‚Š‚é‚Æ‚«‚É‚Ç‚ê‚­‚ç‚¢‚Ì•„†‚̃YƒŒ‚ª¶‚¶‚é‚©‚ð‹³‚Š‚Ä‚­‚ê‚é.
%%%%%%%%%%%%%%%%%%%%%%%%%%%%%%%%%%%%%%%%
\newpage
%%%%%%%%%%%%%%%%%%%%%%%%%%%%%%%%%%%%%%%%
\begin{theorem*}[•œ•ûè—]‚Ì‘ŠŒÝ–@‘¥]
$p,q$ ‚𑊈قȂéŠï‘f”‚Æ‚·‚é.‚±‚Ì‚Æ‚«, ŽŸ‚ª¬‚è—§‚Â:
\[
\left(\frac{p}{q}\right)=(-1)^{\frac{p-1}{2}\frac{q-1}{2}}\left(\frac{q}{p}\right).
\]
\end{theorem*}
\begin{proof}
\[
S:=\left\{ 1,2,\cdots,\frac{p-1}{2}\right\}
,\ T:=\left\{ 1,2,\cdots,\frac{q-1}{2}\right\}
\]
‚Æ‚š‚­.
$s\in S$ ‚ɑ΂µ‚Ä,
$qs\equiv e_s(q)s_{q} \mod p$ ‚æ‚èŽOŠpŠÖ”‚ÌŽüŠú«‚ð—˜—p‚·‚é‚Æ,
\[
\sin\left(\frac{2\pi s}{p}q\right)
=
e_{s}(q)\sin\left(\frac{2\pi s_q }{p}\right)
\]
‚Æ‚È‚é.
‘Ήž $S\ni s\longmapsto s_q\in S$ ‚Ì‘S’PŽË«‚ÆGauss‚Ì•â‘è‚©‚ç
\[
\left( \frac{q}{p}\right)=\prod_{s\in S}e_{s}(q)
=\prod_{s\in S}\left(\sin\left(\frac{2\pi s}{p}q\right)\Big/\sin\left(\frac{2\pi s_q }{p}\right)\right)
=\prod_{s\in S}\left(\sin\left(\frac{2\pi s}{p}q\right)\Big/\sin\left(\frac{2\pi s }{p}\right)\right)
\]
‚ª‚í‚©‚é.ŽOŠp–@‚Ì•â‘è‚ð“K—p‚·‚ê‚Î
\begin{align*}
\prod_{s\in S}\left(\sin\left(\frac{2\pi s}{p}q\right)\Big/\sin\left(\frac{2\pi s }{p}\right)\right)
&=
\prod_{s\in S}(-4)^{\frac{q-1}{2}}\prod_{t\in T}\left(\sin^2\left(\frac{2\pi s }{p}\right)-\sin^2\left(\frac{2\pi t}{q}\right)\right)
\\
&=
(-4)^{\frac{q-1}{2}\frac{p-1}{2}}\prod_{s\in S,t\in T}\left(\sin^2\left(\frac{2\pi s }{p}\right)-\sin^2\left(\frac{2\pi t}{q}\right)\right)
\end{align*}
\[
\therefore \left( \frac{q}{p}\right)=(-4)^{\frac{q-1}{2}\frac{p-1}{2}}\prod_{s\in S,t\in T}\left(\sin^2\left(\frac{2\pi s }{p}\right)-\sin^2\left(\frac{2\pi t}{q}\right)\right).
\]
‚Ü‚œ, ‹c˜_‚̑Ώ̐«‚©‚ç $p,q, S,T$ ‚Ì—§ê‚ð“ü‚ê‘Ö‚Š‚邱‚Æ‚Å,
\[
\left( \frac{p}{q}\right)=(-4)^{\frac{p-1}{2}\frac{q-1}{2}}\prod_{s\in S,t\in T}\left(\sin^2\left(\frac{2\pi t }{q}\right)-\sin^2\left(\frac{2\pi s}{p}\right)\right).
\]
‚ª‚í‚©‚é.Ï‚Ì’†g‚ð $A-B$ ‚Æ‚¢‚€Œ`‚©‚ç‚©‚ç $B-A$ ‚É’Œ‚·‚±‚Æ‚Å
\begin{align*}
\left( \frac{p}{q}\right)
&=
(-4)^{\frac{p-1}{2}\frac{q-1}{2}}\prod_{s\in S,t\in T}\left(\sin^2\left(\frac{2\pi t }{q}\right)-\sin^2\left(\frac{2\pi s}{p}\right)\right)
\\&=
(-1)^{\frac{p-1}{2}\frac{q-1}{2}}(-4)^{\frac{q-1}{2}\frac{p-1}{2}}\prod_{s\in S,t\in T}\left(\sin^2\left(\frac{2\pi s }{p}\right)-\sin^2\left(\frac{2\pi t}{q}\right)\right)
\\&=
(-1)^{\frac{p-1}{2}\frac{q-1}{2}}\left( \frac{q}{p} \right).
\end{align*}
ˆÈã‚Å‚í‚ê‚í‚ê‚ÌŽŠ‚µ‚œ‚©‚Á‚œ‚±‚Æ‚ª‚·‚ׂďؖŸ‚³‚ê‚œ.
\end{proof}
\end{document}
